\documentclass[12pt, a4paper]{article}
\usepackage[utf8]{inputenc}
\usepackage[T1]{fontenc}
\usepackage{amsmath}
\usepackage{amssymb}
\usepackage{geometry}

\geometry{
    a4paper,
    total={170mm,257mm},
    left=20mm,
    top=20mm,
}

\title{\textbf{Fenômenos de Transporte II}}
\author{Gustavo Pimentel da C. Correia}
\date{Julho de 2025}

\begin{document}

\maketitle
\thispagestyle{empty}

\tableofcontents
\newpage

\section{Introdução à Transferência de Calor}

\subsection{Lei de Fourier (Condução)}
\begin{equation}
    q = -k \frac{dT}{dx} 
\end{equation}
Forma vetorial:
\begin{equation}
    \underline{q} = -k \underline{\nabla}T
\end{equation}

\subsection{Lei de Resfriamento de Newton (Convecção)}
\begin{equation}
    q = h (T_{sup} - T_{\infty})
\end{equation}

\subsection{Lei de Stefan-Boltzmann (Radiação)}
Poder emissivo de um corpo negro:
\begin{equation}
    E_B = \sigma T^4
\end{equation}
Poder emissivo de uma superfície real:
\begin{equation}
    E = \epsilon \sigma T^4
\end{equation}
Taxa de transferência líquida para uma superfície convexa cinza em vizinhança grande:
\begin{equation}
    q_{rad} = \epsilon \sigma (T_{sup}^4 - T_{viz}^4)
\end{equation}
Coeficiente de transferência radiante:
\begin{equation}
    h_{rad} = \epsilon \sigma (T_{sup}^2 + T_{viz}^2)(T_{sup} + T_{viz})
\end{equation}

\subsection{Balanço de Energia}
Forma geral para um volume de controle:
\begin{equation}
    \dot{E}_{E} - \dot{E}_{S} + \dot{E}_{G} = \dot{E}_{AC}
\end{equation}
Em regime permanente:
\begin{equation}
    \dot{E}_{E} - \dot{E}_{S} + \dot{E}_{G} = 0
\end{equation}

\newpage

\section{Equação da Condução de Calor}

\subsection{Formas da Equação da Condução}
Forma geral em termos de fluxo (vetorial):
\begin{equation}
    \rho C_p \frac{\partial T}{\partial t} = -\underline{\nabla} \cdot \underline{q} + G
\end{equation}
Com a Lei de Fourier ($k$ pode variar com a posição):
\begin{equation}
    \rho C_p \frac{\partial T}{\partial t} = \underline{\nabla} \cdot (k \underline{\nabla} T) + G
\end{equation}
Com condutividade térmica constante ($k$):
\begin{equation}
    \frac{\partial T}{\partial t} = \alpha \nabla^2 T + \frac{G}{\rho C_p} \quad \text{(Equação de Fourier-Biot)}
\end{equation}
Onde a difusividade térmica $\alpha = \frac{k}{\rho C_p}$.

\subsection{Casos Particulares (com $k$ constante)}
Regime permanente com geração (Equação de Poisson):
\begin{equation}
    \nabla^2 T + \frac{G}{k} = 0
\end{equation}
Regime transiente sem geração (Equação da Difusão):
\begin{equation}
    \frac{\partial T}{\partial t} = \alpha \nabla^2 T
\end{equation}
Regime permanente sem geração (Equação de Laplace):
\begin{equation}
    \nabla^2 T = 0
\end{equation}

\subsection{Condições de Contorno}
\begin{itemize}
    \item \textbf{1º Tipo (Dirichlet):} Temperatura especificada.
    \begin{equation*}
        T(x=0) = T_0
    \end{equation*}
    \item \textbf{2º Tipo (Neumann):} Fluxo de calor especificado.
    \begin{equation*}
        -k \left. \frac{\partial T}{\partial x} \right|_{x=0} = q_0 \implies \left. \frac{\partial T}{\partial x} \right|_{x=0} = -\frac{q_0}{k}
    \end{equation*}
    \item \textbf{3º Tipo (Robin):} Convecção na superfície.
    \begin{equation*}
        -k \left. \frac{\partial T}{\partial x} \right|_{x=L} = h (T|_{x=L} - T_{\infty})
    \end{equation*}
\end{itemize}

\newpage

\section{Condução em Regime Permanente}

\subsection{Resistências Térmicas}
Resistência condutiva (placa plana):
\begin{equation}
    R_{cond} = \frac{L}{kA}
\end{equation}
Resistência condutiva (casca cilíndrica):
\begin{equation}
    R_{cond} = \frac{\ln(R_2/R_1)}{2 \pi L k}
\end{equation}
Resistência convectiva:
\begin{equation}
    R_{conv} = \frac{1}{hA}
\end{equation}

\subsection{Aletas (Extensões Superficiais)}
Equação geral da aleta (1D):
\begin{equation}
    \frac{d^2T}{dx^2} + \frac{1}{A_c}\frac{dA_c}{dx}\frac{dT}{dx} - \frac{hP}{kA_c}(T-T_\infty) = 0
\end{equation}
Para aleta de seção reta uniforme ($A_c$ constante):
\begin{equation}
    \frac{d^2\theta}{dx^2} - m^2\theta = 0
\end{equation}
Onde $\theta = T - T_\infty$ e $m = \sqrt{\frac{hP}{kA_c}}$.

\subsection{Taxa de Calor para Aletas de Seção Reta Uniforme}
Ponta adiabática:
\begin{equation}
    Q_F = \sqrt{hPkA_c} \theta_0 \tanh(mL)
\end{equation}
Aleta infinita:
\begin{equation}
    Q_F = \sqrt{hPkA_c} \theta_0
\end{equation}
Com convecção na ponta:
\begin{equation}
    Q_F = \sqrt{hPkA_c}\theta_0 \frac{\sinh(mL) + (h/mk)\cosh(mL)}{\cosh(mL) + (h/mk)\sinh(mL)}
\end{equation}

\subsection{Desempenho de Aletas}
Efetividade da aleta:
\begin{equation}
    \varepsilon_F = \frac{Q_F}{h A_{c,B} \theta_0}
\end{equation}
Eficiência da aleta:
\begin{equation}
    \eta_F = \frac{Q_F}{h A_F \theta_0}
\end{equation}
Eficiência total da superfície aletada:
\begin{equation}
    \eta_T = 1 - \frac{A_F}{A_T}(1 - \eta_F)
\end{equation}
Resistência térmica total de uma superfície aletada:
\begin{equation}
    R_{term,T} = \frac{1}{\eta_T h A_T}
\end{equation}

\newpage

\section{Condução Transiente}

\subsection{Método dos Parâmetros Concentrados (Bi < 0.1)}
Equação do balanço de energia:
\begin{equation}
    \rho V C_p \frac{dT}{dt} = -hA(T-T_\infty)
\end{equation}
Solução para a temperatura:
\begin{equation}
    \frac{T(t) - T_\infty}{T_i - T_\infty} = \exp\left(-\frac{hA}{\rho V C_p}t\right) = \exp\left(-\frac{t}{\tau}\right)
\end{equation}
Número de Biot e comprimento característico:
\begin{equation}
    Bi = \frac{h L_c}{k} \qquad L_c = \frac{V}{A}
\end{equation}

\subsection{Sólido Semi-Infinito}
Temperatura de superfície constante ($T_s$):
\begin{equation}
    \frac{T(x,t) - T_s}{T_i - T_s} = \text{erf}\left(\frac{x}{2\sqrt{\alpha t}}\right)
\end{equation}
Fluxo de calor na superfície:
\begin{equation}
    q_s(t) = \frac{k(T_s - T_i)}{\sqrt{\pi \alpha t}}
\end{equation}

\subsection{Soluções para Placa, Cilindro e Esfera (Fo > 0.2)}
Aproximação do primeiro termo da série:
\begin{itemize}
    \item \textbf{Placa Plana:} $\theta^*(x^*, Fo) = c_1 \cos(\lambda_1 x^*) \exp(-\lambda_1^2 Fo)$
    \item \textbf{Cilindro Infinito:} $\theta^*(r^*, Fo) = c_1 J_0(\lambda_1 r^*) \exp(-\lambda_1^2 Fo)$
    \item \textbf{Esfera:} $\theta^*(r^*, Fo) = c_1 \frac{\sin(\lambda_1 r^*)}{\lambda_1 r^*} \exp(-\lambda_1^2 Fo)$
\end{itemize}
Onde $\theta^* = \frac{T - T_\infty}{T_i - T_\infty}$, $Fo = \frac{\alpha t}{L^2}$ ou $\frac{\alpha t}{R^2}$, e $c_1, \lambda_1$ dependem do número de Biot.

\subsection{Transferência Multidimensional}
Para um corpo bidimensional (ex: cilindro curto):
\begin{equation}
    \theta^*_{2D}(r, y, t) = \theta^*_{placa}(y, t) \times \theta^*_{cilindro}(r, t)
\end{equation}
Para um paralelepípedo (3D):
\begin{equation}
    \theta^*_{3D}(x,y,z,t) = \theta^*_{placa,x}(x,t) \times \theta^*_{placa,y}(y,t) \times \theta^*_{placa,z}(z,t)
\end{equation}

\newpage

\section{Trocadores de Calor}

\subsection{Coeficiente Global de Transferência de Calor (U)}
Definição geral:
\begin{equation}
    \frac{1}{UA} = R_{eq} = \sum R_{term}
\end{equation}
Para parede cilíndrica (base na área externa):
\begin{equation}
    U_e = \frac{1}{\frac{1}{h_i}\left(\frac{D_e}{D_i}\right) + \frac{D_e \ln(D_e/D_i)}{2k} + \frac{1}{h_e}}
\end{equation}

\subsection{Método da Média Logarítmica da Diferença de Temperatura (LMTD)}
Taxa de calor:
\begin{equation}
    Q = UA \Delta T_{LM}
\end{equation}
Média logarítmica da diferença de temperatura:
\begin{equation}
    \Delta T_{LM} = \frac{\Delta T_1 - \Delta T_2}{\ln(\Delta T_1 / \Delta T_2)} = \frac{(T_{h,i} - T_{c,o}) - (T_{h,o} - T_{c,i})}{\ln\left(\frac{T_{h,i} - T_{c,o}}{T_{h,o} - T_{c,i}}\right)} \quad \text{(Contracorrente)}
\end{equation}
Com fator de correção F para outras configurações:
\begin{equation}
    Q = U A F \Delta T_{LM, cc}
\end{equation}
Balanços de energia:
\begin{align*}
    Q &= \dot{m}_h C_{p,h} (T_{h,i} - T_{h,o}) \\
    Q &= \dot{m}_c C_{p,c} (T_{c,o} - T_{c,i})
\end{align*}

\subsection{Método da Efetividade (ε-NUT)}
Efetividade ($\varepsilon$):
\begin{equation}
    \varepsilon = \frac{Q}{Q_{max}} = \frac{C_h(T_{h,i} - T_{h,o})}{C_{min}(T_{h,i} - T_{c,i})} = \frac{C_c(T_{c,o} - T_{c,i})}{C_{min}(T_{h,i} - T_{c,i})}
\end{equation}
Número de Unidades de Transferência (NUT):
\begin{equation}
    NUT = \frac{UA}{C_{min}}
\end{equation}
Relação para trocador contracorrente:
\begin{equation}
    \varepsilon = \frac{1 - \exp[-NUT(1-C_R)]}{1 - C_R \exp[-NUT(1-C_R)]}
\end{equation}
Onde $C_R = C_{min}/C_{max}$.

\subsection{Deposição (Fouling)}
Resistência de depósito ($R_f$):
\begin{equation}
    \frac{1}{U_d A} = \frac{1}{U_c A} + \frac{R_{f,T}}{A} \implies R_{f,T} = \frac{1}{U_d} - \frac{1}{U_c}
\end{equation}

\newpage

\section{Introdução à Convecção}
\textit{(Este capítulo é majoritariamente conceitual, introduzindo os parâmetros adimensionais)}

\subsection{Parâmetros Adimensionais Fundamentais}
Número de Reynolds (forças de inércia / forças viscosas):
\begin{equation}
    Re_L = \frac{\rho v L}{\mu} = \frac{vL}{\nu}
\end{equation}
Número de Prandtl (difusividade de momento / difusividade térmica):
\begin{equation}
    Pr = \frac{C_p \mu}{k} = \frac{\nu}{\alpha}
\end{equation}
Número de Nusselt (transferência de calor convectiva / condutiva):
\begin{equation}
    Nu_L = \frac{hL}{k}
\end{equation}

\section{Equação da Energia}
\textit{(Este capítulo detalha a derivação da Equação da Energia)}

\subsection{Equação da Energia para Fluido Incompressível}
Forma vetorial (propriedades constantes):
\begin{equation}
    \rho C_p \frac{DT}{Dt} = k \nabla^2 T + G
\end{equation}
Onde a derivada material é $\frac{D}{Dt} = \frac{\partial}{\partial t} + \underline{v} \cdot \underline{\nabla}$.

\newpage

\section{Convecção Forçada em Escoamento Externo}

\subsection{Placa Plana - Escoamento Laminar ($Re_x < 5 \cdot 10^5$)}
Número de Nusselt local ($Pr > 0.6$):
\begin{equation}
    Nu_x = \frac{h_x x}{k} = 0.332 Re_x^{1/2} Pr^{1/3}
\end{equation}
Número de Nusselt médio:
\begin{equation}
    \overline{Nu}_L = \frac{\overline{h}L}{k} = 0.664 Re_L^{1/2} Pr^{1/3}
\end{equation}

\subsection{Placa Plana - Escoamento Turbulento}
Número de Nusselt local ($0.6 < Pr < 60$):
\begin{equation}
    Nu_x = 0.0296 Re_x^{4/5} Pr^{1/3}
\end{equation}
Número de Nusselt médio (regime misto, $Re_{x,c} = 5 \cdot 10^5$):
\begin{equation}
    \overline{Nu}_L = (0.037 Re_L^{4/5} - 871) Pr^{1/3}
\end{equation}

\subsection{Cilindro e Esfera}
Cilindro (Churchill e Bernstein):
\begin{equation}
    \overline{Nu}_D = 0.3 + \frac{0.62 Re_D^{1/2} Pr^{1/3}}{[1+(0.4/Pr)^{2/3}]^{1/4}} \left[1 + \left(\frac{Re_D}{282000}\right)^{5/8}\right]^{4/5}
\end{equation}
Esfera (Whitaker):
\begin{equation}
    \overline{Nu}_D = 2 + (0.4 Re_D^{1/2} + 0.06 Re_D^{2/3})Pr^{0.4}\left(\frac{\mu}{\mu_s}\right)^{1/4}
\end{equation}

\newpage

\section{Convecção Forçada em Escoamento Interno}

\subsection{Escoamento Laminar ($Re_D < 2300$)}
Fator de atrito (desenvolvido):
\begin{equation}
    C_f = \frac{16}{Re_D}
\end{equation}
Número de Nusselt (desenvolvido):
\begin{align*}
    Nu_D &= 3.66 \quad (\text{Temperatura de superfície uniforme}) \\
    Nu_D &= 4.36 \quad (\text{Fluxo de calor uniforme})
\end{align*}
Entrada térmica (Hausen, $Pr \ge 5$, $T_s$ uniforme):
\begin{equation}
    \overline{Nu}_D = 3.66 + \frac{0.0668 Gz_D}{1+0.04 Gz_D^{2/3}}
\end{equation}
Onde $Gz_D = Re_D Pr (D/L)$ é o número de Graetz.

\subsection{Escoamento Turbulento ($Re_D \ge 10000$)}
Correlação de Dittus-Boelter (escoamento desenvolvido):
\begin{equation}
    Nu_D = 0.023 Re_D^{4/5} Pr^n
\end{equation}
onde $n=0.4$ para aquecimento e $n=0.3$ para resfriamento.

\subsection{Tubos não Circulares}
Diâmetro hidráulico:
\begin{equation}
    D_H = \frac{4 A_c}{P_w}
\end{equation}
Para região anular:
\begin{equation}
    D_H = D_{t,i} - d_{t,e}
\end{equation}
Número de Reynolds em função da vazão mássica por tubo:
\begin{equation}
    Re_D = \frac{4 \dot{m}}{\pi \mu D}
\end{equation}

\newpage

\section{Convecção Natural}

\subsection{Parâmetros Adimensionais}
Número de Grashof (Forças de empuxo / Forças viscosas):
\begin{equation}
    Gr_L = \frac{g \beta (T_s - T_\infty) L^3}{\nu^2}
\end{equation}
Para um gás ideal, $\beta = 1/T$.
Número de Rayleigh:
\begin{equation}
    Ra_L = Gr_L Pr = \frac{g \beta (T_s - T_\infty) L^3}{\nu \alpha}
\end{equation}

\subsection{Placa Vertical (Transição em $Ra_L \approx 10^9$)}
Correlação geral (Churchill e Chu):
\begin{equation}
    \overline{Nu}_L = \left\{ 0.825 + \frac{0.387 Ra_L^{1/6}}{[1+(0.492/Pr)^{9/16}]^{8/27}} \right\}^2
\end{equation}

\subsection{Cilindro Horizontal e Esfera}
Cilindro horizontal (Churchill e Chu):
\begin{equation}
    \overline{Nu}_D = \left\{ 0.60 + \frac{0.387 Ra_D^{1/6}}{[1+(0.559/Pr)^{9/16}]^{8/27}} \right\}^2
\end{equation}
Esfera (Churchill):
\begin{equation}
    \overline{Nu}_D = 2 + \frac{0.589 Ra_D^{1/4}}{[1+(0.469/Pr)^{9/16}]^{4/9}}
\end{equation}

\subsection{Convecção Combinada (Natural e Forçada)}
\begin{equation}
    Nu^n = Nu_{for}^n \pm Nu_{nat}^n
\end{equation}
O sinal é '+' para escoamento assistido/transversal e '-' para escoamento oposto.

\newpage

\section{Convecção com Mudança de Fase: Condensação}

\subsection{Filme Laminar em Placa Vertical}
Coeficiente de convecção médio:
\begin{equation}
    \overline{h}_L = 0.943 \left[ \frac{\rho_l (\rho_l - \rho_v) g k_l^3 \lambda}{\mu_l (T_{sat} - T_s)L} \right]^{1/4}
\end{equation}
Número de Reynolds para o filme:
\begin{equation}
    Re_\delta = \frac{4 \dot{m}}{\mu_l w}
\end{equation}
Regimes: Laminar ($Re_\delta \le 30$), com ondas ($30 < Re_\delta < 1800$), Turbulento ($Re_\delta \ge 1800$).

\subsection{Cilindro Horizontal}
\begin{equation}
    \overline{h}_D = 0.729 \left[ \frac{\rho_l (\rho_l - \rho_v) g k_l^3 \lambda}{\mu_l (T_{sat} - T_s)D} \right]^{1/4}
\end{equation}

\subsection{Feixe de Tubos Horizontais}
Coeficiente médio para N tubos na vertical:
\begin{equation}
    \overline{h}_{D,N} = \overline{h}_D N^{-1/4}
\end{equation}
Onde $\overline{h}_D$ é o coeficiente para um único tubo.

\newpage

\section{Radiação Térmica}

\subsection{Radiação de Corpo Negro}
Poder emissivo (Lei de Stefan-Boltzmann):
\begin{equation}
    E_B = \sigma T^4
\end{equation}
Poder emissivo espectral (Distribuição de Planck):
\begin{equation}
    E_{\lambda,B}(\lambda, T) = \frac{C_1}{\lambda^5[\exp(C_2/\lambda T) - 1]}
\end{equation}
Lei do Deslocamento de Wien:
\begin{equation}
    \lambda_{max} T = 2898 \, \mu\text{m} \cdot \text{K}
\end{equation}

\subsection{Propriedades de Superfícies}
Emissividade ($\epsilon$), absortividade ($\alpha$), refletividade ($\rho$), transmissividade ($\tau$):
\begin{equation}
    \alpha + \rho + \tau = 1
\end{equation}
Para superfícies opacas ($\tau=0$):
\begin{equation}
    \alpha + \rho = 1
\end{equation}
Lei de Kirchhoff (superfície cinza, difusa):
\begin{equation}
    \epsilon = \alpha
\end{equation}
Radiosidade (J): Energia que deixa a superfície (emitida + refletida)
\begin{equation}
    J = E + \rho G = \epsilon E_B + \rho G
\end{equation}

\subsection{Troca de Radiação entre Superfícies Cinzas e Difusas}
Taxa de calor líquida para a superfície $i$:
\begin{equation}
    Q_i = \frac{E_{B,i} - J_i}{(1-\epsilon_i) / (\epsilon_i A_i)} \quad \text{(Resistência Superficial)}
\end{equation}
Taxa de calor líquida entre as superfícies $i$ e $j$:
\begin{equation}
    Q_{ij} = \frac{J_i - J_j}{1 / (A_i F_{ij})} \quad \text{(Resistência Espacial)}
\end{equation}
Balanço de energia no nó da radiosidade $J_i$:
\begin{equation}
    \frac{E_{B,i} - J_i}{(1-\epsilon_i) / (\epsilon_i A_i)} = \sum_{j=1}^{N} \frac{J_i - J_j}{1 / (A_i F_{ij})}
\end{equation}
Para uma cavidade com duas superfícies:
\begin{equation}
    Q_{12} = \frac{\sigma(T_1^4 - T_2^4)}{\frac{1-\epsilon_1}{\epsilon_1 A_1} + \frac{1}{A_1 F_{12}} + \frac{1-\epsilon_2}{\epsilon_2 A_2}}
\end{equation}

\end{document}